A summary of the project-specific loading criteria follows (see appendix A for a detailed list of load values).

\subsection{Gravity loading}
The gravity loads listed in Table \ref{grav_load} are in addition to the self weight of the structure. The minimum loading requirements were taken from ASCE 7 as well as the loading criteria supplied by the engineer of record. Loads are given in pounds per square foot (psf).

\begin{table}[h]
  \begin{center}
    \begin{tabular}{lll}
      \textbf{Use} & \textbf{Live Loading} & \textbf{Superimposed} \\
      &&\textbf{Dead Loading} \\
      \hlineB{2}
      Parking Garage & 40 & 3 \\
      \arrayrulecolor{gray}\hline
      Storage/HVAC & 125 & 28 \\
      \arrayrulecolor{gray}\hline
      Stairways, exits & 100 & 28 \\
      \arrayrulecolor{gray}\hline
      Level 1 residential & 40 & 28 \\
      \arrayrulecolor{gray}\hline
      Level 1 corridors & 100 & 28 \\
      \arrayrulecolor{gray}\hline
      Level 1 office, recreational & 100 & 28 \\
      \arrayrulecolor{gray}\hline
      Level 1 courtyard (footprint) & 150 & 150 \\
      \arrayrulecolor{gray}\hline
      Elevated levels residential & 40 & 28 \\
      \arrayrulecolor{gray}\hline
      Elevated levels corridors & 40 & 28 \\
      \arrayrulecolor{gray}\hline
      Cornices & 60 & - \\
      \arrayrulecolor{gray}\hline
      Balconies & 40 & 28 \\
      \arrayrulecolor{gray}\hline
      Roof & 20 & 28 \\
      \hlineB{2}
  \end{tabular}
  \caption{Gravity Loads} \label{grav_load}
  \end{center}
\end{table}

In addition to these uniform slab loads, a perimeter dead load of 12 psf was applied to the structure to account for the weight of the cladding system.

\subsection{Wind design criteria}
Wind loading is in accordance with the IBC and ASCE 7 requirements as shown in Table \ref{wind_load}.
\begin{table}[h]
  \begin{center}
    \begin{tabular}{ll}
      \textbf{Parameter} & \textbf{Value} \\
      \hlineB{2}
Basic Wind Speed, 3-second gust (ultimate) & 115 mph \\
      \arrayrulecolor{gray}\hline
Basic Wind Speed, 3-second gust (nominal) & 90 mph \\
      \arrayrulecolor{gray}\hline
Exposure & B \\
      \arrayrulecolor{gray}\hline
Occupancy Category & II \\
      \arrayrulecolor{gray}\hline
Importance Factor ($I_w$ ) & 1.0 \\
      \arrayrulecolor{gray}\hline
Topographic Factor ($K_{zt}$ ) & 1.0\\
      \arrayrulecolor{gray}\hline
Enclosure Classification & Enclosed \\
      \arrayrulecolor{gray}\hline
Mean Roof Height (h) & 33' \\
      \hlineB{2}
  \end{tabular}
  \caption{Wind Design Criteria} \label{wind_load}
  \end{center}
\end{table}

\subsection{Snow loading}
Wind loading is in accordance with the ASCE 7 requirements as shown in Table \ref{snow_load}.
\begin{table}[h]
  \begin{center}
    \begin{tabular}{ll}
      \textbf{Parameter} & \textbf{Value} \\
      \hlineB{2}
Ground snow load $p_g$ & 60 psf \\ 
      \arrayrulecolor{gray}\hline
Terrain category & B \\
      \arrayrulecolor{gray}\hline
Exposure factor $C_e$ &  1.0 \\
      \arrayrulecolor{gray}\hline
Thermal factor $C_t$ & 1.0 \\
      \arrayrulecolor{gray}\hline
Occupancy Category & II \\
      \arrayrulecolor{gray}\hline
Snow load importance factor $I_s$ & 1.0\\
      \arrayrulecolor{gray}\hline
Snow load flat roof & 42 psf \\ 
      \hlineB{2}
  \end{tabular}
  \caption{Snow Design Criteria} \label{snow_load}
  \end{center}
\end{table}

\section{Seismic design criteria}
Seismic loads are in accordance with the IBC requirements as shown in Table \ref{seism_load}.
\begin{table}[h]
  \begin{center}
    \begin{tabular}{ll}
      \textbf{Parameter} & \textbf{Value} \\
      \hlineB{2}
Building Latitude/Longitude & 44$^o$49'01.8"N 91$^o$30'34.8"W \\
Occupancy Category & II\\
Importance Factor $I_e$ &  1.0\\
Mapped Spectral Acceleration & $S_s$ = 0.045; $S_1$ = 0.038 \\
Site Class & B \\
Site Class Coefficients & $F_a$ = 1.0; $F_v$ = 1.0 \\
Spectral Response Coefficients & $S_{DS}$ = 0.03; $S_{D1}$ = 0.025 \\
Seismic Design Category & A \\
      \hlineB{2}
  \end{tabular}
  \caption{Snow Design Criteria} \label{seism_load}
  \end{center}
\end{table}

\clearpage

\section{Dead loads}
\begin{table}[h!]
\begin{tabular}{lp{8cm}p{10cm}}
\multicolumn{3}{l}{Materials}\\
& Wood structural panel  & $ 36.0\ \mathrm{pcf} = 5655 \cfrac{\mathrm{newton}}{\mathrm{meter}^3}$ \\
&  Concrete reinforced stone (including gravel)  & $ 150.0\ \mathrm{pcf} = 23563 \cfrac{\mathrm{newton}}{\mathrm{meter}^3}$ \\
& Steel  & $ 489.0\ \mathrm{pcf} = 76816 \cfrac{\mathrm{newton}}{\mathrm{meter}^3}$ \\
& Gypsum crete  & $ 115.0\ \mathrm{pcf} = 18065 \cfrac{\mathrm{newton}}{\mathrm{meter}^3}$ \\

& Gypsum,loose  & $ 70.0\ \mathrm{pcf} = 10996 \cfrac{\mathrm{newton}}{\mathrm{meter}^3}$ \\
& Earth (not submerged) sand and gravel (wet)  & $ 120.0\ \mathrm{pcf} = 18850 \cfrac{\mathrm{newton}}{\mathrm{meter}^3}$ \\
& Water  & $ 62.4\ \mathrm{pcf} = 9802 \cfrac{\mathrm{newton}}{\mathrm{meter}^3}$ \\


\multicolumn{3}{l}{Frame partitions}\\
& Wood or steel studs, $\frac{1}{2}$in gypsum board inside & $8\ \mathrm{psf}= 383 \ \mathrm{pascal}$ \\
& Wood studs, 2x4 unplastered & $4\ \mathrm{psf}=192 \ \mathrm{pascal}$\\
& Wood studs, 2x4 plastered one side & $12\ \mathrm{psf}= 575\ \mathrm{pascal}$\\
& Wood studs, 2x4 plastered two sides & $20\ \mathrm{psf}=958 \ \mathrm{pascal}$\\
& Movable steel partitions &  $4\ \mathrm{psf}=192 \ \mathrm{pascal}$\\
\multicolumn{3}{l}{Frame walls}\\
& Exterior stud wall 2x4 @ 16in, $\frac{5}{8}$ gypsum insulated, $\cfrac{3}{8}$ in siding &  $11\ \mathrm{psf}=526 \ \mathrm{pascal}$\\
& Exterior stud wall 2x6 @ 16in, $\frac{5}{8}$ gypsum insulated, $\cfrac{3}{8}$ in siding &  $12\ \mathrm{psf}=575 \ \mathrm{pascal}$\\
& Exterior stud wall with brick veneer &   $48\ \mathrm{psf}=2298 \ \mathrm{pascal}$\\
& CMU wall 8in &   $60\ \mathrm{psf}=9425 \ \mathrm{pascal}$\\
& Window, glass, frame and sash & $8\ \mathrm{psf}=383 \ \mathrm{pascal}$\\
\multicolumn{3}{l}{Cladding}\\
& Fiber cement panels, large format $38.4\text{in} \times 102\text{in}$ & $3.2\ \mathrm{psf} = 153\ \mathrm{pascal}$ \\
& Fiber cement panels, small scale $9.6\text{in} \times 102\text{in}$ & $3.2\ \mathrm{psf} = 153\ \mathrm{pascal}$ \\
& Perforated metal panel at exterior HVAC location & \\
\multicolumn{3}{l}{Floor truss}\\
& Single chord @ 24in o.c. spacing & $3.2\ \mathrm{psf} = 153\ \mathrm{pascal}$ \\
& Double chord @ 24in o.c. spacing & $4.25\ \mathrm{psf} = 203\ \mathrm{pascal}$ \\
\multicolumn{3}{l}{Sheating}\\
& Roof sheating & $ 3.5\ \mathrm{psf} = 167\ \mathrm{pascal}$ \\
& Floor sheating & $ 2.5\ \mathrm{psf} = 120\ \mathrm{pascal}$ \\
& Ceilings & $ 2.5\ \mathrm{psf} = 120\ \mathrm{pascal}$ \\
& Deck composite sleeperes (3in) &  $9.00\ \mathrm{psf} = 431\ \mathrm{pascal}$ \\
\end{tabular}
\end{table}

\section{Live loads}
\begin{tabular}{p{5cm}p{2.5cm}|p{2.5cm}|p{4cm}}
\textbf{Occupancy or use} & \textbf{Uniform} & \textbf{Concentrated} & \textbf{Notes}\\
\hline
Private rooms and corridors serving them in multifamily dwelling & $ 40.0\ \mathrm{psf} = 1915 \ \mathrm{pascal}$ & - & \emph{IBC-2018 Table 1607.1}  \\
Stairs and exits & $ 100.0\ \mathrm{psf} = 4788 \ \mathrm{pascal}$ & $300\ \mathrm{pound} = 1334\ \mathrm{newton}$ & \emph{IBC-2018 Table 1607.1. Concentrated load on stair treads applied on an area of 2 inches by 2 inches}  \\
Balconies and decks & same as occupancy served & - &  \emph{IBC-2018 Table 1607.1}  \\
Garages (passenger vehicles only) & $ 40.0\ \mathrm{psf} = 1915 \ \mathrm{pascal}$ & - & \emph{IBC-2018 Table 1607.1}  \\
Cornices & $ 60.0\ \mathrm{psf} = 2873 \ \mathrm{pascal}$ & - & \emph{IBC-2018 Table 1607.1}  \\
Elevator machine room and control room grating & - & $300\ \mathrm{pound} = 1334\ \mathrm{newton}$ & \emph{IBC-2018 Table 1607.1.   Concentrated load  applied on an area of 2 inches by 2 inches}\\
Flat roof (not occupiable) + maintenace & $ 20.0\ \mathrm{psf} = 958 \ \mathrm{pascal}$ & $300\ \mathrm{pound} = 1334\ \mathrm{newton}$ & \emph{IBC-2018 Table 1607.1}  \\
Yards and terraces, pedestrians & $ 100.0\ \mathrm{psf} = 4788 \ \mathrm{pascal}$ & - & \emph{IBC-2018 Table 1607.1}  \\
Sidewalks, vehicular driveways and yards, subject to trucking & $ 250.0\ \mathrm{psf} = 11970 \ \mathrm{pascal}$ & $8000\ \mathrm{pound} = 35586\ \mathrm{newton}$ & \emph{IBC-2018 Table 1607.1}  \\
Corridors first floor & $ 100.0\ \mathrm{psf} = 4788 \ \mathrm{pascal}$ & - & \emph{IBC-2018 Table 1607.1}  \\
Store first floor & $ 100.0\ \mathrm{psf} = 4788 \ \mathrm{pascal}$ & - & \emph{IBC-2018 Table 1607.1}  \\

\end{tabular}

\section{Snow loads}
\begin{tabular}{p{5cm}p{5cm}|p{5cm}}
Ground snow load & $p_g = 60.0\ \mathrm{psf} = 2873 \ \mathrm{pascal}$ & \emph{ASCE 7. Figure 7.1}\\
Exposure factor & $C_e = 1.0$ &  \emph{ASCE 7. Table 7-2. Terrain category B, roof partially exposed} \\
Thermal factor &  $C_t = 1.0$ & \emph{ASCE 7. Table 7-3.}\\
Snow load importance factor & $I_s = 1.0$ & \emph{ASCE 7. Table 7-4. Structure risk category II}\\
\textbf{Snow load flat roof} & $p_f = 0.7 \times C_e \times C_t \times  I_s\times p_g = 0.7 \times 1.0 \times 1.0 \times 1.0 \times 60.0 = 42.0\ \mathrm{psf} = 2873 \ \mathrm{pascal}$ & \emph{ASCE 7. Sect. 7.3}\\
\end{tabular}

\section{Wind loads}
\begin{tabular}{p{5cm}l|p{5cm}}
\multicolumn{2}{l|}{Alternate all-heights method.} & \emph{IBC-2018, sect. 1609.6.  Regularly shaped building, less than 75 feet in height, not sensitive to dynamic effects, not channeling effects or buffeting, simple diaphragm building} \\
Ultimate design wind speed & $V_{ult} = 115 \frac{\mathrm{miles}}{\mathrm{hour}} = 51 \frac{\mathrm{meters}}{\mathrm{second}} $& \emph{IBC-2018, figure 1609.3(1). Risk category II building}\\
Velocity pressure exposure coefficient & $K_z = 0.72$ & \emph{ASCE 7, table 27.3.1. Exposure B, height above ground level z $\approx$ 33 feet} \\
Topographic factor & $K_{zt} = 1.0 $ & \emph{ASCE 7, sect. 26.8} \\
\end{tabular}

\vspace{5mm}

\begin{tabular}{lcc|p{5cm}}
\multicolumn{3}{p{8cm}|}{\textbf{Net pressure coefficients $C_{net}$}. Main windforce-resisting frames and systems} & \emph{IBC-2018, Table 1609.6.2, enclosed} \\
Description & $C_{net}$ + Internal & $C_{net}$ - Internal & \\
 & pressure &  presure & \\
 \cline{1-3}
Windward wall & 0.43 & 0.73 & \\
Leeward wall & -0.51 & -0.21 & \\
Sidewall & -0.66 & -0.35 & \\
Parapet windward wall & \multicolumn{2}{c|}{1.28} & \\
Parapet leeward wall & \multicolumn{2}{c|}{-0.85} & \\
Flat roof & -1.09 & -0.79 & \\
\end{tabular}

\vspace{5mm}

\begin{tabular}{lp{4cm}p{4cm}|p{2cm}}
\multicolumn{3}{l|}{\textbf{Design wind pressures $P_{net}$}. Main windforce-resisting frames and systems} & \emph{IBC-2018, sect. 1609.6.3} \\
\multicolumn{3}{p{8cm}|}{$P_{net} = 0.00256 \times V^2 \times K_z \times C_{net} \times K_{zt}$} & \\
Description & $P_{net}$ + Internal & $P_{net}$ - Internal & \\
 & pressure &  presure & \\
 \cline{1-3}
Windward wall & $10.5\ \mathrm{psf} =  501\ \mathrm{pascal}$ & $17.8\ \mathrm{psf} =  852\ \mathrm{pascal}$  & \\
Leeward wall & $-12.4\ \mathrm{psf} =  -595\ \mathrm{pascal}$  & $-5.1\ \mathrm{psf} =  -245\ \mathrm{pascal}$  & \\
Sidewall & $-16.1\ \mathrm{psf} =  -770\ \mathrm{pascal}$  & $-8.5\ \mathrm{psf} =  -409\ \mathrm{pascal}$  & \\
Parapet windward wall & \multicolumn{2}{c|}{$31.2\ \mathrm{psf} =  1494\ \mathrm{pascal}$ } & \\
Parapet leeward wall & \multicolumn{2}{c|}{$-20.7\ \mathrm{psf} =  -992\ \mathrm{pascal}$ } & \\
Flat roof & $-26.6\ \mathrm{psf} =  -1272\ \mathrm{pascal}$  & $-19.3\ \mathrm{psf} =  -992\ \mathrm{pascal}$  & \\
\end{tabular}

\section{Earthquake loads}
\begin{tabular}{p{4cm}l|p{4cm}}
Parameter 0.2-second spectral response acceleration & $S_s = 0.045$ &  \emph{IBC-2018, figure 1613.3.1(1). Site class B} \\
Parameter 1-second spectral response acceleration & $S_1 = 0.038$ &  \emph{IBC-2018, figure 1613.3.1(2). Site class B} \\
Seismic design category & $S_1 \le 0.04\ and\  S_s \le 0.15 \rightarrow $ SDS A & \emph{IBC-2018, sect. 1613.3.1} \\
Site coefficients & $F_a$ = 1.0, $F_v$ = 1.0 & \emph{IBC-2018, tables 1613.3.3(1) and 1613.3.3(2). Site class  B} \\
Maximum considered earthquake spectral response acceleration for short periods & $S_{MS} = F_a\cdot S_s = 0.045$ &  \emph{IBC-2018, sect. 163.3.3} \\
& $S_{M1} = F_a\cdot S_1 = 0.038$ &  \emph{IBC-2018, sect. 163.3.3} \\
Design spectral response acceleration parameters & $S_{DS} = \cfrac{2}{3} S_{MS} = 0.03$ & \emph{IBC-2018, sect. 163.3.4} \\
& $S_{D1} = \cfrac{2}{3} S_{M1} = 0.025$ & \emph{IBC-2018, sect. 163.3.4} \\
\end{tabular}

\end{document}

\end{document}

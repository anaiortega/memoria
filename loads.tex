A summary of the project-specific loading criteria follows (see appendix A for a detailed list of load values).

\subsection{Gravity loading}
The gravity loads listed in Table \ref{grav_load} are in addition to the self weight of the structure. The minimum loading requirements were taken from ASCE 7 as well as the loading criteria supplied by the engineer of record. Loads are given in pounds per square foot (psf).

\begin{table}[h]
  \begin{center}
   \caption{\textbf{Gravity Loads}} \label{grav_load}
   \begin{tabular}{lll}
      \textbf{Use} & \textbf{Live Loading} & \textbf{Superimposed} \\
      &&\textbf{Dead Loading} \\
      \hlineB{2}
      Parking Garage & 40 & 3 \\
      \arrayrulecolor{gray}\hline
      Storage/HVAC & 125 & 28 \\
      \arrayrulecolor{gray}\hline
      Stairways, exits & 100 & 28 \\
      \arrayrulecolor{gray}\hline
      Level 1 residential & 40 & 28 \\
      \arrayrulecolor{gray}\hline
      Level 1 corridors & 100 & 28 \\
      \arrayrulecolor{gray}\hline
      Level 1 office, recreational & 100 & 28 \\
      \arrayrulecolor{gray}\hline
      Level 1 courtyard (footprint) & 150 & 150 \\
      \arrayrulecolor{gray}\hline
      Elevated levels residential & 40 & 28 \\
      \arrayrulecolor{gray}\hline
      Elevated levels corridors & 40 & 28 \\
      \arrayrulecolor{gray}\hline
      Cornices & 60 & - \\
      \arrayrulecolor{gray}\hline
      Balconies & 40 & 28 \\
      \arrayrulecolor{gray}\hline
      Roof & 20 & 28 \\
      \hlineB{2}
  \end{tabular}
  \end{center}
\end{table}

In addition to these uniform slab loads, a perimeter dead load of 12 psf was applied to the structure to account for the weight of the cladding system.

\subsection{Wind design criteria}
Wind loading is in accordance with the IBC and ASCE 7 requirements as shown in Table \ref{wind_load}.
\begin{table}[h]
  \begin{center}
  \caption{\textbf{Wind Design Criteria}} \label{wind_load}
    \begin{tabular}{ll}
      \textbf{Parameter} & \textbf{Value} \\
      \hlineB{2}
Basic Wind Speed, 3-second gust (ultimate) & 115 mph \\
      \arrayrulecolor{gray}\hline
Basic Wind Speed, 3-second gust (nominal) & 90 mph \\
      \arrayrulecolor{gray}\hline
Exposure & B \\
      \arrayrulecolor{gray}\hline
Occupancy Category & II \\
      \arrayrulecolor{gray}\hline
Importance Factor ($I_w$ ) & 1.0 \\
      \arrayrulecolor{gray}\hline
Topographic Factor ($K_{zt}$ ) & 1.0\\
      \arrayrulecolor{gray}\hline
Enclosure Classification & Enclosed \\
      \arrayrulecolor{gray}\hline
Mean Roof Height (h) & 33' \\
      \hlineB{2}
  \end{tabular}
  \end{center}
\end{table}

\subsection{Snow loading}
Wind loading is in accordance with the ASCE 7 requirements as shown in Table \ref{snow_load}.
\begin{table}[h]
  \begin{center}
  \caption{\textbf{Snow Design Criteria}} \label{snow_load}
    \begin{tabular}{ll}
      \textbf{Parameter} & \textbf{Value} \\
      \hlineB{2}
Ground snow load $p_g$ & 60 psf \\ 
      \arrayrulecolor{gray}\hline
Terrain category & B \\
      \arrayrulecolor{gray}\hline
Exposure factor $C_e$ &  1.0 \\
      \arrayrulecolor{gray}\hline
Thermal factor $C_t$ & 1.0 \\
      \arrayrulecolor{gray}\hline
Occupancy Category & II \\
      \arrayrulecolor{gray}\hline
Snow load importance factor $I_s$ & 1.0\\
      \arrayrulecolor{gray}\hline
Snow load flat roof & 42 psf \\ 
      \hlineB{2}
  \end{tabular}
  \end{center}
\end{table}

\section{Seismic design criteria}
Seismic loads are in accordance with the IBC requirements as shown in Table \ref{seism_load}.
\begin{table}[h]
  \begin{center}
  \caption{\textbf{Seismic Design Criteria}} \label{seism_load}
    \begin{tabular}{ll}
      \textbf{Parameter} & \textbf{Value} \\
      \hlineB{2}
Building Latitude/Longitude & 44$^o$49'01.8"N 91$^o$30'34.8"W \\
Occupancy Category & II\\
Importance Factor $I_e$ &  1.0\\
Mapped Spectral Acceleration & $S_s$ = 0.045; $S_1$ = 0.038 \\
Site Class & B \\
Site Class Coefficients & $F_a$ = 1.0; $F_v$ = 1.0 \\
Spectral Response Coefficients & $S_{DS}$ = 0.03; $S_{D1}$ = 0.025 \\
Seismic Design Category & A \\
      \hlineB{2}
  \end{tabular}
  \end{center}
\end{table}


